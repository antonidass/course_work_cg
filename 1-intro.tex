\newpage
\chapter*{Введение}
\addcontentsline{toc}{chapter}{Введение}

В наши дни научные деятели активно изучают космические пространства и объекты. Ученые стремятся максимально реалистично изобразить моделируемые системы, поэтому сегодня важно иметь инструменты для разработки реалистичных трехмерных изображений. На помощь приходит компьютерная графика.

Компьютерная графика - наука, представляющая собой совокупность методов и способов преобразования информации в графическое представление при помощи ЭВМ. Без компьютерной графики не обходится ни одна современная программа. В течении нескольких десятилетий компьютерная графика прошла долгий путь, начиная с базовых алгоритмов, таких как вычерчивание линий и отрезков, до построения виртуальной реальности.

Целью данного курсового проекта является разработка ПО, которое предоставляет визуализацию космических объектов.

В рамках выполнения работы необходимо решить следующие задачи.

\begin{enumerate}
	\item Описать предметную область работы.
	\item Рассмотреть существующие алгоритмы построения реалистичных изображений.
	\item Выбрать и обосновать выбор реализуемых алгоритмов.
	\item Выбрать способ хранения данных о моделируемом объектею.
	\item Подробно изучить выбранные алгоритмы.
	\item Выбрать среду и язык программирования.	
	\item Разработать программу на основе существующих алгоритмов.
	\item Снизить время работы выбранного алгоритма.
	\item Выбрать и обосновать выбор языка программирования, для решения данной задачи.
\end{enumerate}