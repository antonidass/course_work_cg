\chapter{Аналитическая часть}

В данном разделе рассматриваются существующие алгоритмы построения реалистичных изображений, обосновывается выбор реализуемого алгоритма и указывается список ограничений, в рамках которых будет работать разрабатываемое ПО.

\section{Формализация объектов синтезируемой сцены}
Сцена состоит из следующих объектов:
\begin{itemize}
    \item точечный источник света -- представляет собой материальную точку, испускающую лучи света во все стороны;
    \item задний фон;
    \item космические объекты: ракета и планета -- представляют собой набор полигонов (пример таких объектов представлен на рисунке \ref{img:rocket_planet}).
\end{itemize}

\imgs{rocket_planet}{ht!}{0.7}{Пример ракеты и планеты}

В компьютерной графике в основном используются 3 вида моделей трехмерных объектов:
\begin{itemize}
    \item каркасная (проволочная) модель. Это простейший вид моделей, содержащий минимум информации -- о вершинах и рёбрах объектов. Главный недостаток -- такая модель не всегда правильно передает представление об объекте (например, если в объекте есть отверстия);
    \item поверхностная модель. Отдельные участки задаются как участки поверхности того или иного вида. Эта модель решает проблему каркасной, но все еще имеет недостаток -- нет информации о том, с какой стороны поверхности находится собственный материал;
    \item объемная модель. В отличии от поверхностной, содержит указание расположения материала (чаще всего указанием направления внутренней нормали). 
\end{itemize} 

Важнейшие требования к модели -- правильность отображения информации об объекте и компактность. В рамках поставленной задачи каркасная модель не удовлетворяет первому критерию, а информация о том, где расположен материал, не является необходимой, что делает объемную модель избыточной, поэтому поверхностная модель является наиболее подходящей. 

Существует несколько способов задания поверхностной модели:
\begin{itemize} 
\item аналитический способ. Этот способ задания модели характеризуется описанием модели объекта, которое доступно в неявной форме, то есть для получения визуальных характеристик необходимо дополнительно вычислять некоторую функцию, которая зависит от параметра;

\item полигональная сетка. Данный способ характеризуется совокупностью вершин, граней и ребер, которые определяют форму многогранного объекта в трехмерной компьютерной графике.
\end{itemize}

Стоит отметить, что одним из решающих факторов в выборе способа задания модели в данном проекте является скорость выполнения преобразований над объектами сцены. Поэтому при реализации программного продукта в данной работе наиболее удобным представлением является модель, заданная полигональной сеткой -- это поможет избежать проблем при описании сложных моделей. При этом способ хранения полигональной сетки -- это список граней, так как он содержит явное описание граней, что поможет при реализации алгоритма удаления невидимых рёбер и поверхностей. 


\section{Описание алгоритмов удаления невидимых линий и поверхностей}

\subsection {Некоторые теоретические сведения}

Алгоритмы удаления невидимых линий и поверхностей служат для определения линий ребер, поверхностей, которые видимы или невидимы для наблюдателя, находящегося в заданной точке пространства \cite{rogers}.

Решать задачу можно в:
\begin{itemize}
	\item объектном пространстве -- используется мировая система координат, достигается высокая точность изображения. Обобщенный подход, основанный на анализе пространства объектов, предполагает попарное сравнение положения всех объектов по отношению к наблюдателю.
	\item пространстве изображений -- используется экранная система координат, связанная с устройством в котором отображается результат. (графический дисплей).
\end{itemize}

Под экранированием подразумевается загораживание одного объекта другим.

\subsection{Алгоритм Робертса}

Алгоритм Робертса решает задачу удаления невидимых линий. Работает в объектном пространстве. Данный алгоритм работает исключительно с выпуклыми телами. Если тело изначально является невыпуклым, то нужно его разбить на выпуклые составляющие. Алгоритм целиком основан на математических предпосылках \cite{rogers}.

Из-за сложности математических вычислений, используемых в данном алгоритме, а так же из-за дополнительных затраты ресурсов на вычисление матриц данный алгоритм является довольно медленным.


\subsection{Алгоритм Варнака}

Алгоритм Варнака \cite{rogers,shykin,bayackovskiy} позволяет определить, какие грани или части граней объектов сцены видимы, а какие заслонены гранями других объектов. Так же как и в алгоритме Робертса анализ видимости происходит в пространстве изображения. В качестве граней обычно выступают выпуклые многоугольники, алгоритмы работы с ними эффективнее, чем с произвольными многоугольниками. Окно, в котором необходимо отобразить сцену, должно быть прямоугольным. Алгоритм работает рекурсивно, на каждом шаге анализируется видимость граней и, если нельзя легко определить видимость, окно делится на 4 части и анализ повторяется отдельно для каждой из частей (см. рис. \ref{img:Warnock_1}).

\imgs{Warnock_1}{ht!}{0.7}{Пример разбиения Алгоритмом Варнока}

Так как данный алгоритм основывается на рекурсивном разбиении экрана, в зависимости от расположения объектов это может вызвать, как положительной, так и отрицательный эффект. Чем меньше пересечений объектов, тем быстрее алгоритм завершит свою работу.

\subsection{Алгоритм Z-буфера}

Алгоритм Z-буфера \cite{rogers,shykin} позволяет определить, какие пикселы граней сцены видимы, а какие заслонены гранями других объектов. Z-буфер -- это двухмерный массив, его размеры равны размерам окна, таким образом, каждому пикселу окна, соответствует ячейка Z-буфера. В этой ячейке хранится значение глубины пиксела (см. рис. \ref{img:z-buff}). Перед растеризацией сцены Z-буфер заполняется значением, соответствующим максимальной глубине. В случае, когда глубина характеризуется значением w,
максимальной глубине соответствует нулевое значение. Анализ видимости происходит при растеризации граней, для каждого пиксела рассчитывается глубина и сравнивается со значением в Z-буфере, если рисуемый пиксел ближе (его \texttt{w} больше значения в Z-буфере), то пиксел рисуется, а значение
в Z-буфере заменяется его глубиной. Если пиксел дальше, то пиксел не рисуется и Z-буфер не изменяется, текущий пиксел дальше того, что нарисован ранее, а значит невидим. 

\imgs{z-buff}{h!}{0.7}{Пример работы алгоритма Z-буфера}

К недостаткам алгоритма следует отнести довольно большие объемы требуемой памяти, а также имеются другие недостатки, которые состоят в трудоемкости устранения лестничного эффекта и трудности реализации эффектов прозрачности.

\subsection{Алгоритм прямой трассировки лучей}

Основная идея алгоритма прямой трассировки лучей \cite{shykin} состоит в том, что наблюдатель видит объекты благодаря световым лучам, испускаемым некоторым источником, которые падают на объект, отражаются, преломляются или проходят сквозь него и в результате достигают зрителя.

Основным недостатком алгоритма является излишне большое число рассматриваемых лучей, приводящее к существенным затратам вычислительных мощностей, так как лишь малая часть лучей достигает точки наблюдения. Данный алгоритм подходит для генерации статических сцен и моделирования
зеркального отражения, а так же других оптических эффектов \cite{traceproblem}.

\subsection{Алгоритм обратной трассировки лучей}

Алгоритм обратной трассировки лучей отслеживает лучи в обратном направлении (от наблюдателя к объекту) \cite{shykin}. Такой подход призван повысить эффективность алгоритма в сравнении с алгоритмом прямой трассировки лучей. Обратная трассировка позволяет работать с несколькими источниками света, передавать множество разных оптических явлений \cite{snizko}.
 
Пример работы данного алгоритма приведен на рисунке \ref{img:trace_scheme}.

\imgs{trace_scheme}{h!}{0.7}{Пример работы алгоритма обратной трассировки лучей}

Считается, что наблюдатель расположен на положительной полуоси z в бесконечности, поэтому все световые лучи параллельны оси z. В ходе работы испускаются лучи от наблюдателя и ищутся пересечения луча и всех объектов сцены \cite{bayackovskiy}. В результате пересечение с максимальным значением z является видимой частью поверхности и атрибуты данного объекта используются для определения характеристик пиксела, через центр которого проходит данный световой луч. 

Для расчета эффектов освещения сцены проводятся вторичные лучи от точек пересечения ко всем источникам света. Если на пути этих лучей встречается непрозрачное тело, значит, данная точка находится в тени.

Несмотря на более высокую эффективность алгоритма в сравнении с прямой трассировкой лучей, данный алгоритм считается достаточно медленным, так как в нем происходит точный расчет сложных аналитических выражений для нахождения пересечения с рассматриваемыми объектами.



\subsection{Алгоритм художника}
Данный алгоритм работает аналогично тому, как художник рисует картину – то есть сначала рисуются дальние объекты, а затем более близкие. Пример работы данного алгоритма приведен на рисунке \ref{img:hudozhnik}.  
Наиболее распространенная реализация алгоритма – сортировка по глубине, которая заключается в том, что произвольное множество граней сортируется по ближнему расстоянию от наблюдателя, а затем отсортированные грани выводятся на экран в порядке от самой дальней до самой ближней. Данный метод работает лучше для построения сцен, в которых отсутствуют пересекающиеся грани~\cite{rogers}.
\imgs{hudozhnik}{h!}{0.2}{Пример работы алгоритма художника}

Основным недостатком алгоритма является высокая сложность реализации при пересечении граней на сцене.


\section{Анализ и выбор модели освещения}

Физические модели материалов стараются аппроксимировать свойства некоторого реального материала. Такие модели учитывают особенности поверхности материала или же поведение частиц материала.

Эмпирические модели материалов устроены иначе, чем физически обоснованные. Данные модели подразумевают некий набор параметров, которые не имеют физической интерпретации, но которые позволяют с помощью подбора получить нужный вид модели.

В данной работе следует делать выбор из эмпирических моделей, а конкретно из модели Ламберта и модели Фонга.

\subsection{Модель Ламберта}

Модель Ламберта моделирует идеальное диффузное освещение, то есть свет при попадании на поверхность рассеивается равномерно во все стороны. При такой модели освещения учитывается только ориентация поверхности ($N$) и направление источника света ($L$). Иллюстрация данной модели представлена на рисунке \ref{img:mod_lam}.

\imgs{mod_lam}{h!}{1}{Направленность источника света}

Эта модель является одной из самых простых моделей освещения и очень часто используется в комбинации с другими моделями. Она может быть очень удобна для анализа свойств других моделей, за счет того, что ее легко выделить из любой модели и анализировать оставшиеся составляющие.

\subsection{Модель Фонга}

Это классическая модель освещения. Модель представляет собой комбинацию диффузной и зеркальной составляющих. Работает модель таким образом, что кроме равномерного освещения на материале могут появляться блики. Местонахождение блика на объекте определяется из закона равенства углов падения и отражения. Чем ближе наблюдатель к углам отражения, тем выше яркость соответствующей точки.

\imgs{mod_fong}{h!}{1}{Направленность источника света}

Падающий и отраженный лучи лежат в одной плоскости с нормалью к отражающей поверхности в точке падения (рисунок \ref{img:mod_fong}). Нормаль делит угол между лучами на две равные части. $L$ – направление источника света, $R$ – направление отраженного луча, $V$ – направление на наблюдателя.


\section{Вывод}

Поскольку в данной моделируемой системе космические объекты не пересекаются, для  удаления невидимых линий был выбран алгоритм художника. Данный алгоритм позволит добиться максимальной производительности, что особенно важно для динамических сцен. Стоит отметить тот факт, что алгоритм художника не требователен к памяти, в отличие, например, от алгоритма Z-буфера.

В качестве модели освещения в данной работе была выбрана модель Ламберта из-за своей простоты по сравнению с моделью Фонга. Для расчета данных модели Ламберта необходимо выполнять меньше вычислений, а значит ее реализация потребует меньшего количества времени.


