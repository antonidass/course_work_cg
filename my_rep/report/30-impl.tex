\chapter{Технологическая часть}

В данном разделе будут представлены средства разработки программного обеспечения, интерфейс программного обеспечения и процесс сборки разрабатываемого программного обеспечения.

\section{Выбор средств реализации}

В качестве языка программирования для разработки программного обеспечения был выбран язык \texttt{Java} \cite{cpp}. Данный выбор обусловлен тем, что данный язык предоставляет весь функционал требуемый для решения поставленной задачи.

Для создания пользовательского интерфейса ПО был использован фреймворк \texttt{AWT} \cite{qt}. Данный фреймворк содержит в себе объекты, позволяющие напрямую работать с пикселами изображения, а  так же возможности создания интерактивных пользовательских интерфейсов, что позволит в интерактивном режиме управлять изображением.

Для сборки программного обеспечения использовался инструмент -- \texttt{Gradle}.

В качестве среды разработки был выбран текстовый редактор \texttt{IntelliJ IDEA} \cite{vscode}, поддерживающий возможность установки плагинов, в том числе для работы с \texttt{Java} и \texttt{Gradle}.


\section{Описание процесса сборки приложения}

Действия, необходимые для сборки проекта приведены приведены в листинге \ref{lst:build}:

\listingfile{build.sh}{build}{bash}{Сборка реализуемого программного обеспечения}{}

В результате выполнения данных действий будет скомпилирован jar файл.  



\clearpage
\section{Интерфейс ПО}

Интерфейс реализуемого ПО представлен на Рисунках \ref{img:regims} -- \ref{img:speed}.

\imgw{regims}{ht!}{0.6\textwidth}{Интерфейс программы -- группа выбора режима}

На рисунке \ref{img:regims} представлен интерфейс настройки выбора режима, включающий в себя выбор трех режимов: <<реалистичный>>, <<каркас>>, <<только модели>>. Режим <<каркас>> представляет собой отображение сцены с помощью каркасной модели. <<Реалистичный>> режим представляет собой закрашенные модели с задним фоном. Режим <<только модели>> предоставляет собой режим с закрашенными моделями без заднего фона.
\clearpage

\imgw{vid}{ht!}{0.6\textwidth}{Интерфейс программы -- группа настроек камеры}

На рисунке \ref{img:vid} представлен интерфейс настройки параметров камеры (точки наблюдения), включающий в себя выбор положения точки наблюдения: <<сверху>>, <<снизу>>, <<спереди>>

\clearpage
\imgw{lights}{ht!}{0.6\textwidth}{Интерфейс программы -- группа настроек освещения сцены}

На рисунке \ref{img:lights} представлен интерфейс настройки параметров освещения синтезируемой сцены, включающий в себя выбор положения источника освещения, а также выбор количества источников освещения. 

\clearpage
\imgw{speed}{ht!}{0.6\textwidth}{Интерфейс программы -- группа настроек скорости системы}

На рисунке \ref{img:speed} представлен интерфейс настройки скорости системы.

\section{Вывод}

В данном разделе были представлены средства разработки программного обеспечения, детали реализации и процесс сборки разрабатываемого программного обеспечения. 