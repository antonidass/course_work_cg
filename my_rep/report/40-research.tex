\chapter{Экспериментальная часть}

В данном разделе будет поставлен эксперимент, в котором будут сравнены временные характеристики работы реализованного программного обеспечения в различных конфигурациях.

\section{Цель эксперимента}

Целью эксперимента является проверка правильности выполнения поставленной задачи, оценка эффективности при увелечении количества отображаемых объектов.

\section{Апробация}

На рисунках \ref{img:tkarkas} -- \ref{img:treal} представлены результаты синтеза изображения в различных режимах. Можно заметить, что сцены работают корректно.

\imgw{tkarkas}{ht!}{\textwidth}{Визуализация сцены в режиме <<Каркас>>}

\imgw{tmodel}{ht!}{\textwidth}{Визуализация сцены в режиме <<Только модели>>}

\imgw{treal}{ht!}{\textwidth}{Визуализация сцены в режиме <<Реалистичный>>}


\clearpage
На рисунках \ref{img:tall} -- \ref{img:tniz} приведен результат работы программы при разном положении наблюдателя. Виды сверху, спереди и снизу работают корректно.

\imgw{tall}{ht!}{\textwidth}{Вид спереди, добавлены все источники освещения}
\clearpage

\imgw{tverh}{ht!}{\textwidth}{Вид сверху}
\clearpage

\imgw{tniz}{ht!}{\textwidth}{Вид снизу}

\section{Технические характеристики}

Технические характеристики устройства, на котором выполнялось исследование:

\begin{itemize}
	\item процессор: Intel Core™ i5-8250U \cite{i5} CPU @ 1.60GHz;
	\item память: 16 GiB;
	\item операционная система: Manjaro \cite{manjaro} Linux \cite{linux} 21.1.4 64-bit.
\end{itemize}

Исследование проводилось на ноутбуке, включенном в сеть электропитания. Во время тестирования ноутбук был нагружен только встроенными приложениями окружения рабочего стола, окружением рабочего стола, а также непосредственно системой тестирования.

\section{Описание эксперимента}
Предметом серии поставленных экспериментов является реализованный в рамках курсового проекта алгоритм художника.

В рамках данного эксперимента будет производиться оценка влияния количества изображаемых объектов на время работы алгоритма. Для этого будем синтезировать сцены с количествами объектов равными [300, 400, 500, \dots, 1000].
 
Для снижения погрешности измерений будем усреднять получаемые значения. Для этого каждое из измерений будет проводиться $N = 100$ раз, после чего будет вычисляться среднее арифметическое значение измеряемой величины.

\section{Результат эксперимента}

На рисунке \ref{plt:time} приведены графики зависимости времени синтеза изображения от количества изображаемых объектов.

\begin{figure}[ht]
	\centering
	\begin{tikzpicture}
		\begin{axis}[
			axis lines=left,
			xlabel={Количество объектов},
			ylabel={Время, мс},
			legend pos=north west,
			ymajorgrids=true
		]
			\addplot table[x=size,y=1,col sep=comma] {inc/csv/time.csv};
		\end{axis}
	\end{tikzpicture}
	\captionsetup{justification=centering}
	\caption{График зависимости времени синтеза изображения от количества изображаемых объектов}
	\label{plt:time}
\end{figure}

Из рисунка \ref{plt:time} следует, что реализованный алгоритм художника зависит линейно от количества объектов. При этом интерактивность реализуемого ПО теряется при количестве изображаемых объектов равном 500 и более.  

\section{Вывод}

В данном разделе было произведено экспериментально сравнение временных характеристик реализованного программного обеспечения.

Время работы алгоритма имеет линейную зависимость от количества отображаемых объектов.
