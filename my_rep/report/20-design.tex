\chapter{Конструкторская часть}

В данном разделе будут рассмотрены требования к программному обеспечению, а также схемы алгоритмов, выбранных для решения поставленной задачи. Так же, будут описаны пользовательские структуры данных и приведена структура реализуемого программного обеспечения.

\section{Требования к программному обеспечению}

ПО должно предоставлять доступ к следующему функционалу:

\begin{itemize}
    \item визуальное отображение сцены;
    \item изменение скорости движения системы космических объектов;
    \item выбор режима отображения моделей;
    \item перемещение камеры (точки наблюдения);
    \item изменение положения источника света;
    \item изменение количества источников света.
\end{itemize}

К ПО предъявляются следующие требования:

\begin{itemize}
    \item время отклика программы не должно превышать 5 секунд для корректной работы в интерактивном режиме;
    \item программа должна корректно реагировать на любые действия пользователя.
\end{itemize}


\section{Алгоритм Художника}
Этот метод работает аналогично тому, как художник рисует картину -- сначала рисуются более далекие объекты, а затем близкие. Метод основывается на том факте, что если самая дальняя точка грани $A$ ближе к наблюдателю, чем самая ближняя точка грани $B$, то грань $B$ никак не может закрыть грань $A$. Однако, в некоторых случаях это не всегда так. Можно подобрать такие две грани $A$ и $B$, что для них не будет выполняться это правило. Тогда стоит применить следующую модификацию данного алгоритма: берутся средние значения расстояний от вершин граней до наблюдателя и для двух граней $A$ и $B$ выполняется сравнение этих значений. Если среднее значение глубины для грани $A$ меньше соответствующего значения для грани $B$, то грань $A$ закрывает грань $B$ и сначала рисуется грань $B$, а затем -- грань $A$. Наиболее распространенная реализация алгоритма сортировки по глубине заключается в том, что произвольное множество граней сортируется по ближайшему расстоянию до наблюдателя, а затем отсортированные грани выводятся на экран в порядке от самой дальней к самой ближней. 

\section{Модель освещения Ламберта}

Данная модель вычисляет цвет поверхности в зависимости от того как на нее светит источник света. Согласно данной модели, освещенность точки равна произведению силы источника света и косинуса угла, под которым он светит на точку.

\begin{equation}
	\label{eq:lambert}
	I = I_0 * cos(L, N),
\end{equation}
где:
\begin{itemize}
	\item $I$ -- результирующая интенсивность света в точке;
	\item $I_0$ -- интенсивность источника;
	\item $L$ -- направление из точки на источник;
	\item $N$ -- вектор нормали.
\end{itemize}

На Рисунке \ref{img:algos} представлена схема синтеза изображения с применением алгоритма Художника.
\clearpage

\imgs{algos}{ht!}{0.65}{Схема алгоритма синтеза изображения с применением алгоритма Художника}

\section{Представление данных в ПО}

В данной работе используются следующие типы и структуры данных:

\begin{itemize}
    \item источник света -- задается расположением;
	\item точка трехмерного пространства -- хранит направление по x, y, z;
	\item цвет -- хранит три составляющие \texttt{RGB} модели цвета;
	\item объект сцены -- список полигонов.
\end{itemize}

\section{Описание структуры программного обеспечения}

На Рисунке \ref{img:class} представлена диаграмма классов реализуемого программного обеспечения.

\imgs{class}{ht!}{0.75}{Диаграмма классов реализуемого ПО}


\section{Вывод}

В данном разделе были представлены требования к разрабатываемому программному обеспечению и разработана схема разрабатываемого алгоритма. Так же, были описаны пользовательские структуры данных и приведена структура реализуемого программного обеспечения.